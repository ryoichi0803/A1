\section{考察}
\subsection{寿命について}
求まった寿命$\tau= 2.19 \pm 0.08\,\mathrm{\mu s}$は文献値とよく一致している。
このフィットの$\chi^2 / \mathrm{ndf}$も十分1に近く、精度よく実験できたと考えられる。

\subsection{$g$因子について}



\subsection{反省:MPPCについて}
例年の測定ではPMTのみを使った測定だったが、今回はPMTとMPPCとを併用した測定となった。初めての試みだったのでいくつかの反省点、今期以降使う方への留意点を上げていく。
\begin{itemize}
\item セッティング
\end{itemize}
MPPCはダイオードであるので、取り付ける向きを間違えると大電流が流れ、壊れる原因になってしまう。今回の測定でも一つ向きを間違えてしまったものが見つかったが、取り替えをせず向きを訂正して使用してしまった。また、素子が回路から外れやすく、遮光等のセッティングの際にも注意すべきである。
\begin{itemize}
\item Breakdown電圧の調整
\end{itemize}
MPPCの型番によってBreakdown電圧に差があり、各MPPCに対するデータはあるにしろ、基盤によって変化してしまうこともあり‎揃えるのに非常に時間がかかってしまった。また、温度が上がるとBreakdown電圧が上がるという温度依存性も考慮する必要があり、細心の注意を払わなければならない。
