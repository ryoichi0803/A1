\section{考察}
\subsection{寿命について}
求まった寿命$\tau= 2.19 \pm 0.08\,\mathrm{\mu s}$は文献値とよく一致している。
このフィットの$\chi^2 / \mathrm{ndf}$も十分1に近く、精度よく実験できたと考えられる。

\subsection{$g$因子について}
本実験において得られたデータでは、正確なg因子の値を求めるにはデータ数が不足していると考えられる。
ここではその理由について説明する。
\subsubsection{ビン数の問題}
1000nsから20000nsをいくつのビンに分けるかを考える。

ビンの数を多くするにつれ、各ビンに属するデータの数は少なくなる。
本実験の実験bでは1000nsから20000nsに、MPPCで1930カウント、PMT1で1435カウントしかなかった。
例えばビン数を100にすると、1000ns付近のビンでも100カウントほどのイベントしかない。
\ref{eq:g-fit}におけるスピン歳差運動による"寿命のゆらぎ"は、文献値のg因子と、測定した磁場の値から予想される$\omega$を使って計算すると全体のイベント数の10\%以下になることがわかるが、これは先の例だと数カウントほどしかヒストグラムに影響されないことになる。
ヒストグラムをみてわかるように、スピン歳差運動に起因するものではないと考えらるゆれが多くあるので、これをフィットするのは難しい。

一方、$\omega = 4.0 \times 10^{-3}\,\mathrm{1/ns}$ほどのオーダーであることを考慮すると、\ref{eq:g-fit}の"ゆらぎ"の周期はおよそ1500nsほどとなる。
再びビン数を100にすると、このとき一つのビンの幅は190nsになり、このとき一周期はおよそ8つのビンで表されることになるが、この数で一周期を表現するのには、多くのデータが必要である。

以上より、ビンの数を多くしようとしても、少なくしようとしてもフィットに困難が生じる。

\subsubsection{実際のフィット}
実際にビンの数を100にして、\ref{eq:g-fit}を用いてフィットしてみると、
\begin{align}
  A &= 150 \pm 11 \\
  \tau &= 2.100 \pm 0.12\,\mathrm{\mu s} \\
  \omega &= (3.97 \pm 0.23) \times 10^{-3}\,\mathrm{1/ns}\\
  t_0 &= -3974 \pm 320 \mathrm{ns} \\
  B &= 7.73 \pm 0.40
\end{align}
のようになった。
この$\omega$から求められる$g$因子は$1.88 \pm 0.17$となる。
文献値$2.002$を誤差の範囲で含むものの、これは予想される$\omega$の値をフィットの初期条件に選んだことによる可能性が高い。

\subsection{反省:MPPCについて}
例年の測定ではPMTのみを使った測定だったが、今回はPMTとMPPCとを併用した測定となった。初めての試みだったのでいくつかの反省点、今期以降使う方への留意点を上げていく。
\begin{itemize}
\item セッティング
\end{itemize}
MPPCはダイオードであるので、取り付ける向きを間違えると大電流が流れ、壊れる原因になってしまう。今回の測定でも一つ向きを間違えてしまったものが見つかったが、取り替えをせず向きを訂正して使用してしまった。また、素子が回路から外れやすく、遮光等のセッティングの際にも注意すべきである。
\begin{itemize}
\item Breakdown電圧の調整
\end{itemize}
MPPCの型番によってBreakdown電圧に差があり、各MPPCに対するデータはあるにしろ、基盤によって変化してしまうこともあり‎揃えるのに非常に時間がかかってしまった。また、温度が上がるとBreakdown電圧が上がるという温度依存性も考慮する必要があり、細心の注意を払わなければならない。
