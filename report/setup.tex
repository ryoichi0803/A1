実験のセットアップ
$\mu$を検出しアナログ信号に変えるために用いた装置は以下のものからなっている。
・プラスチックシンチレータ($100\times 48\times 1cm$)2枚
・プラスチックシンチレータ($120\times 5cm$)7枚
・光電子増倍管(PMT)2本
・コイル(詳細は後述)1つ
・銅板1つ
・MPPC及びそのための基盤(詳細は後述)7つ
・光ファイバー7本
以上の装置を次の図のように$\mu$の寿命測定(以下実験a)、$\mu$のg因子測定(以下実験b)それぞれ配置した。
実験a
\begin{figure}[h]
  \includegraphics[]{}
\end{figure}
実験b
\begin{figure}[h]
  \includegraphics[]{}
\end{figure}

上記の装置で$\mu$は、以下のような振る舞いを行う。
1.$\mu$がどこからか降って来て、AとBを突き抜け、銅板にたどり着く。
2.銅板にたどり着いた$\mu$はエネルギーが高い場合突き抜け、Cまでたどり着くが、
あるエネルギー程度の$\mu$はCu版で止まる。
3.2で銅板で止まった$\mu$は時間が経過すると弱い相互作用によって崩壊する。
4.3で生じた陽電子がB、またはCにたどり着く。

なお、実験aとbのそれぞれのセットアップでMPPCとPMT0の距離を変えたのは、
aでは$\mu$のカウント数をなるべく増やして統計誤差を減らすために狭く設定したが、
bではシンチレータに対して垂直な向きで降ってくる$\mu$のみを検出するために広く設定した。
実験の回路図
\begin{figure}[h]
  \includegraphics[]{}
\end{figure}
