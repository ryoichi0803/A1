\section{実験のセットアップ}
\subsection{セットアップ・及び実験原理}
$\mu$を検出しアナログ信号に変えるために用いた装置は以下のものからなっている。
・プラスチックシンチレータ($100\times 48\times 1cm$)2枚
・プラスチックシンチレータ($50\times 5\times 1cm$)7枚
・光電子増倍管(PMT)2本
・コイル(詳細は後述)1つ
・銅板($50\times 48\times 1cm$)1つ
・MPPC及びそのための基盤(詳細は後述)7つ
・光ファイバー7本
以上の装置を次の図のように$\mu$の寿命測定(以下実験a)、$\mu$のg因子測定(以下実験b)それぞれ配置した。
\subsubsection{実験a}
\begin{figure}[h]
  \includegraphics[width=5cm,bb=0 0 640 250]{zikken_a.jpg}
  \caption{$\mu$の寿命測定}
\end{figure}
\subsubsection{実験b}
\begin{figure}[h]
  \includegraphics[width=5cm,bb=0 0 640 400]{zikken_b.jpg}
  \caption{$\mu$のg因子測定装置}
\end{figure}
ここで、A,B,Cはいずれもプラスチックシンチレータである。
上記の装置で$\mu$は、以下のような振る舞いを行う。
1.$\mu$がどこからか降って来て、AとBを突き抜け、銅板にたどり着く。
2.銅板にたどり着いた$\mu$はエネルギーが高い場合突き抜け、Cまでたどり着くが、
あるエネルギー程度の$\mu$はCu板で止まる。
3.2で銅板で止まった$\mu$は時間が経過すると弱い相互作用によって崩壊する。
4.3で生じた陽電子がB、またはCにたどり着く。

なお、実験aとbのそれぞれのセットアップでMPPCとPMT0の距離を変えたのは、aでは$\mu$のカウント数をなるべく増やして統計誤差を減らすために狭く設定したが、bではシンチレータに対してスピンの方向が垂直な向きであることを仮定しているので、あまり上のシンチレータと銅板との間隔が狭いと斜めからの$\mu$が入り込んでしまうことも考慮して、aより広く間隔を設定した。
\subsection{実験の回路図}
\begin{figure}[htbm]
  \includegraphics[width=5cm, bb=0 0 640 500]{kairo.jpg}
  \caption{実験回路図}
\end{figure}
また、この回路図でdelay31nsを組んでいるのは、MPPCとPMT0、PMT1の信号をCOINCIDENCEでタイミングを合わせるためである。
この回路を論理記号を用いて表すと、
\begin{itemize}
  \item Start信号 $(PMT0 \land MPPC)\lnot PMT1 $
  \item Stop信号0~6 $ MPPC 0~6 $
  \item Stop信号7 $ MPPC1 $
\end{itemize}
となる。

\subsection{コイル}
  本実験の\(g\)因子測定において、磁場をコイルから発生させた。コイルは2004年度P1の課題研究用に作成されたものを使用した。このコイルはMainコイルとして4つのコイルを並列でつないだものを並べ、さらに2つのSubコイルを用いたものを使用した。
Mainコイルには31.0Vで19.0A、sub1コイルには3.5Vで0.80A、sub2コイルには3.6Vで0.80Aの電流を流した。
\begin{figure}[htbp]
 \begin{center}
  \includegraphics[width=100mm,bb=0 0 640 400]{coil.jpg}
  \caption{coil.tex}
 \end{center}
\end{figure}
磁場の測定では、コイルに電流を流してから24時間経過して磁場が安定したあと、
測定位置を銅板が入っている部分を銅板の大きさが縦50cm横48cmなので、上図のように16のセクションに分け、それぞれの場所においての磁場を計3回測定した。
その測定結果を下表に記す。
\begin{center}
\begin{tabular}{c|ccc|c}\hline
  セクション&1回目[Gauss]&2回目[Gauss]&3回目[Gauss]&3回の平均[Gauss]\\ \hline
  1.a & 51.0 & 47.0 & 49.4 & \\
  1.b & 53.0 & 50.1 & 53.6 & \\
  1.c & 51.6 & 50.4 & 50.2 & \\
  1.d & 49.6 & 49.7 & 49.6 & \\
  2.a & 54.7 & 49.2 & 52.0 & \\
  2.b & 55.2 & 49.5 & 50.2 & \\
  2.c & 53.2 & 49.0 & 48.7 & \\
  2.d & 50.2 & 49.8 & 49.9 & \\
  3.a & 51.2 & 48.8 & 49.4 & \\
  3.b & 50.2 & 48.9 & 49.2 & \\
  3.c & 49.8 & 48.3 & 50.5 & \\
  3.d & 49.6 & 47.7 & 48.9 & \\
  4.a & 50.6 & 48.0 & 49.3 & \\
  4.b & 49.2 & 49.4 & 49.8 & \\
  4.c & 51.5 & 47.4 & 48.0 & \\
  4.d & 49.8 & 48.4 & 49.3 & \\
 ave.& 51.3 & 48.9 & 49.5 & \\
\end{tabular}
\end{center}
この測定結果から磁場を計算すると、


\subsection{TDCの較正}

\subsection{PMT(光電子増倍管)・及びDiscriminatorの設定}
ここで、$\mu$の寿命測定及びg因子測定で用いたPMTの電圧・及びDiscriminatorの設定の過程を述べる。
\subsubsection{Discriminatorの閾値設定}
まず、本実験で用いたDiscriminatorの閾値の設定方法について述べる。この閾値をうまく設けることでPMT及びMPPCからの信号のうちバックグラウンドのイベントをシャットアウトし、検出したい$\mu$の信号のみを検出することができる。
本実験では、まず最初に、まず最初にMPPCとPMT両方のDiscriminatorの閾値を30mVに設定し、他のノイズをシャットアウトできるように、またgainが揃うようにそれぞれ1700V、1750Vとして測定を開始した。しかし、この測定結果のグラフを見たところ、これはDiscriminatorの閾値が低く、$\mu$以外のバックグラウンドも検出していると判断し、Discriminatorの閾値を決めるために、今度はMPPCの$\mu$の信号をそれぞれオシロスコープで見たところ、ほぼ60mVでそろっていたので、Discriminatorの閾値を59mVで設定しなおした。また、Stop信号がMPPC、PMT1それぞれに来ている割合を見ると、PMT1にほとんど信号が来ていなかったので、これはPMTの電圧設定が低すぎて$\mu$の信号が検出できてないことが原因と判断し、PMTの電圧設定を高くし、PMT0、PMT1それぞれ1949V、1998Vと設定しなおし、バックグラウンドをシャットアウトするために、Discriminatorの閾値を59mVに上げた。
